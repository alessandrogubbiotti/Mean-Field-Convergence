%date 1/12/2015

\documentclass[12pt,a4paper]{article}
\usepackage{amsmath,amsfonts,amssymb,amsthm,amscd,mathrsfs}
\usepackage{stmaryrd}
\usepackage{cases}
\usepackage[english]{babel}
%\usepackage[mathscr]{eucal}
\usepackage[T1]{fontenc}
\usepackage{aeguill}
\usepackage{enumerate}

\usepackage{amsmath,amssymb,graphics,epsfig,color,cite}
\usepackage{mathrsfs,bbm}
\usepackage{amsthm}
\usepackage{amsmath,amssymb,graphics,epsfig,color,
mathrsfs,bbm}


\usepackage{graphics,epsfig,color,cite,bbm}


\definecolor{light}{gray}{.9}

\DeclareMathAlphabet{\mathpzc}{OT1}{pzc}{m}{it}

\linespread{1.1}



\pagestyle{plain}



 

\newcommand{\pecetta}[1]{
$\phantom .$
\bigskip
\par\noindent
\colorbox{light}{\begin{minipage}{12.3cm}#1\end{minipage}}
\bigskip
\par\noindent
}

\newcommand{\mb}[1]{\mathbb{#1}}
\newcommand{\ms}[1]{\mathscr{#1}}
\newcommand{\mf}[1]{\mathfrak{#1}}
\newcommand{\mc}[1]{\mathcal{#1}}
\newcommand{\mbf}[1]{\mathbf{#1}}
\newcommand{\R}[1]{\mb{R}^{#1}}
\newcommand{\im}{\text{i}}

\newcommand{\Lp}{\mc{L}_+}
\newcommand{\Np}{\mc{N}_+}

\newcommand{\fp}{F(M\times V)}
\newcommand{\fps}{F_s(M\times V)}
\newcommand{\fpa}{F_a(M\times V)}
\newcommand{\func}[1]
{F(M\times V^{#1})}
\newcommand{\funca}[1]
{F_a(M\times V^{#1})}
\newcommand{\funcs}[1]
{F_s(M\times V^{#1})}

\newcommand{\funcc}[1]      %continuous case
{F(M_{G}\times G^{#1})}
\newcommand{\funcca}[1]
{F_a(M_{G}\times G^{#1})}
\newcommand{\funccs}[1]
{F_s(M_{G}\times G^{#1})}

\newcommand{\funcv}[1]
{F(M_{G}\times V^{#1})}
\newcommand{\funcva}[1]
{F_a(M_{G}\times V^{#1})}
\newcommand{\funcvs}[1]
{F_s(M_{G}\times V^{#1})}


\DeclareMathOperator{\Dif}{Dif}
\DeclareMathOperator{\Sum}{Sum}
\DeclareMathOperator{\Alt}{Alt}
\DeclareMathOperator{\Sym}{Sym}
\DeclareMathOperator{\dif}{dif}
\DeclareMathOperator{\sign}{sign}
\DeclareMathOperator{\Tay}{Tay}
\DeclareMathOperator{\odd}{odd}
\DeclareMathOperator{\even}{even}
\DeclareMathOperator{\Vol}{Vol}
\DeclareMathOperator{\st}{st}
\DeclareMathOperator{\Cell}{Cell}
\DeclareMathOperator{\cell}{cell}
\DeclareMathOperator{\Card}{Card}
\DeclareMathOperator{\SO}{SO}
\DeclareMathOperator{\cut}{Cut}
\DeclareMathOperator{\supp}{supp}
\DeclareMathOperator{\spec}{Spec}
\DeclareMathOperator{\tr}{Tr}
\DeclareMathOperator{\trace}{tr}
\DeclareMathOperator{\hess}{Hess}
\DeclareMathOperator{\lin}{lin}
\DeclareMathOperator{\symb}{Symb}
\DeclareMathOperator{\End}{End}
\DeclareMathOperator{\homog}{hom}
\DeclareMathOperator{\pr}{pr}
\DeclareMathOperator{\spr}{spr}
\DeclareMathOperator{\id}{Id}
\DeclareMathOperator{\const}{Const}
\DeclareMathOperator{\dist}{dist}
\DeclareMathOperator{\app}{app}
\DeclareMathOperator{\range}{Ran}
\DeclareMathOperator{\ess}{ess}
\DeclareMathOperator{\osc}{osc}
\DeclareMathOperator{\rev}{rev}
\DeclareMathOperator{\inw}{inw}
%\DeclareMathOperator{\dim}{dim}
 
 \def\variance{{\rm Var}}
\def\ent{{\rm Ent\,}}% Hessian


 
 \def\Diag{{\rm Diag\,}}% Diagonal Matrix

\def\sp{{\rm Sp\,}}

\def \Sp{{\rm \, Sp\;}}




\newcommand{\arcsinh}
{\operatorname{arcsinh}}  % it s the same!
 
 

\newcommand{\lp}{\langle}
\newcommand{\rp}{\rangle}
\newcommand{\ra}{\rightarrow}
\newcommand{\ve}{\varepsilon}
\newcommand{\vrho}{\varrho}
\newcommand{\vp}{\varphi}

 

\newtheorem{theorem}{Theorem}[section]
\newtheorem{proposition}[theorem]{Proposition}
\newtheorem{definition}[theorem]{Definition}
\newtheorem{lemma}[theorem]{Lemma}
\newtheorem{corollary}[theorem]{Corollary}
\newtheorem{remark}[theorem]{Remark}
\newtheorem{example}[theorem]{Example}
\newtheorem{assumption}{Assumption}[part]


 
 
 




  


\date{\today}          

\begin{document}

%:
 

%\begin{abstract}
%\end{abstract}

 
\

\section{Setting}
\noindent
We fix a smooth bounded connected set  $\Omega\subset \mb R^d$, a function $V\in C^2(\overline\Omega; \mb R)$
and consider the differential operator
\[     L_0    \   :=    \       \frac 12 \Delta \  -  \   \nabla V \cdot \nabla      \   .    \]
\pecetta{More generally one may take
\[     L_0    \   :=    \       \frac 12 \nabla \cdot a \nabla  \  +  \   b  \cdot \nabla      \   ,    \]
or even work with general Markov processes...
}
We denote by $u: [0, \infty)\times \overline{\Omega} \ra \mb R $ the unique solution of 
\[
\begin{cases} 
\partial_t u  \, (t,x) \  =   \     L_0 u  \,  (t,x)              & \text{ for }   (t,x)\in (0, \infty) \times \Omega  \\
u(0,x)   \  =   \   1      & \text{ for }   x \in \Omega\\
u(t,x)   \  =  \   0   & \text{ for }   (t,x) \in (0, \infty) \times \partial \Omega     \  .     \\
\end{cases}    \]
\pecetta{   Probabilistic interpretation:  $u(t,x) \  =  \  \mb P_x(\tau_\Omega>t)$.}
Note that $u>0$ and that $u(t, \cdot)\in C^2 (\Omega;\mb R)$. 
For each $0\leq t \leq T$  let $U_{t, T}:  \Omega \ra \mb R$ be given by 
\[   U_{t,T} (x)  \  :=   \      \log u (T-t, x)  \  ,   \]
and consider the differential operator
\[    L_{t}^T  \  :=  \    L_0    \   +  \   \nabla  U_{t,T}  \cdot \nabla    \  .    \]
For each fixed $T>0$, we denote by $\{\mb P^T_{x}\}_{x\in \Omega}$ the family of Markov distributions  
induced by the time-dependent generator   $ L^T := (L_{t}^T)_{t\in [0,T]}$
on the path space $C([0,T]; \Omega)$. 

\pecetta{Probabilistic interpretation: condition to not be absorbed until time $T$...}


 
\noindent
We define for each  path $X\in C([0,T]; \Omega)$
the empirical measure
\[   \mu_T(X)  \  :=  \    \frac 1T \int_0^T  \delta_{X_t}   \, dt    \  .   \]

\noindent
The goal is to study the asymptotic behaviour when $T\ra \infty$ of the probability distribution
$ P^T_{x}$ on $\mc P (\Omega)$ (the set of probability distributions on $\Omega$)  given by 
\[    P^T_{x}    \  :=   \    \mb P^T_{x} \circ \mu^{-1}_T     \  .     \]

More precisely, we would like to show:

\

{\bf ``Theorem''}


\noindent
 $\{P_{x}^T\}_{T>0}$ satisfies for $T\ra \infty $ a large deviation principle (uniformly for $x$ in a compact ?) with speed
 $T$ and rate function 
 \[   I (\mu)   \  =   \     -    \inf_{f\in C_c^2(\mb R;(0, \infty))}   \int   \frac{L  f}{f} d\mu   \    \  ,    \]
 where 
 \[    L   \  :=  \    L_0    \   +  \   \nabla  \log \vp      \cdot \nabla    \  ,    \]
 and $\vp$ is characterized by $\vp>0$ and  existence of $\lambda>0$ such that
 \[
\begin{cases} 
   - \, L_0 \vp  \,  (x)      \   =   \   \lambda \vp         & \text{ for }   x\in \Omega  \\
\vp(x)   \  =   \   0        & \text{ for }   x \in\partial \Omega     \  .     \\
\end{cases}      \]

\


\noindent
We denote in the sequel by $\{\mb Q^T_{x}\}_{x\in \Omega}$ the family of Markov distributions  
induced by the generator   $ L$ on the path space $C([0,T]; \Omega)$. 
\pecetta{Probabilistic interpretation: process conditioned to never be absorbed,  ``Q-process". }

\begin{remark}It follows from the usual Donsker-Varadhan principle that  $Q^T_{x}:= \mb Q^T_{x} \circ \mu^{-1}_T$
satisfies an LDP with rate $I$ ({\bf reference?}). 
\end{remark}



\

\section{proof}



{\bf Upper bound}

......\pecetta{do usual perturbation with gradient fields, the problem is: time dependent drift, how should the perturbing drift depend on $t, T$?

\

mi sembra ne avevamo discusso a luglio, dovrei riuscire a ricostruirlo.    }

{\bf Lower bound}
........



\pecetta{Instead of doing upper and lower bound, maybe it is easier to show that $Q^T_x$
and $P^T_x$ are exponentially equivalent? 
Something like: let  $\nu, \eta$ random elements in $\mc P(\Omega)$
define on the same probability space and
  having distribution respectively $ P_{x}$ and $Q_{x}$. 
We want to show that for every $\delta>0$
\[    \limsup_{T\ra \infty}  T^{-1} \log  \ms P(|\nu - \eta|>\delta)    \  =   \    -\infty     \   .   \]
}
 
 \begin{thebibliography}{99}



  \end{thebibliography}

 
\end{document}